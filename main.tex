\documentclass{article}
\usepackage[utf8]{inputenc}
\usepackage{chemfig}
\usepackage{layout}
\usepackage{mhchem}
\usepackage{graphicx}
\graphicspath{ {./images/} }

\title{Chem 2}
\author{Etienne Herzig}
\date{21. September 2022}

\begin{document}
\setlength{\textheight}{680pt}

\maketitle
\clearpage
\section{Overview}
\subsection{Grading}
Only the better two tests get counted and make up 80\% of the grade.
Homework \& Assignments should be submitted within online within two weeks and make up 20\% of the grade.
 A bonus of ~4\% can be earned with consistent presence
\subsection{Test Dates}
\date{9.11.2022}\hfill\break
\date{14.12.2022}\hfill\break
\date{25.01.2023}\hfill\break
\clearpage
\section{Lessons}
\subsection{Lesson 1 - 21.09.2022}
Revision of principles learned in Chem 1, which bear relevance in Chem 2. \break
Important things to remember: \hfill
\begin{itemize}
    \item 1. Distinction Atom / Molecule
    \item 2. Polar / Non-Polar
    \begin{itemize}
        \item A molecule is polar if the majority of bond in said molecule are polar. A bond is polar if one Atom in the bond has a higher electronegativity.
        \item Important Polar Bonds
        \begin{itemize}
            \item C - O
            \item O - H
            \item N - H
        \end{itemize}
        \item Important Non-Polar Bonds
        \begin{itemize}
            \item C - C
            \item C - H
        \end{itemize}
        \item Boiling Points are related to the inter-molecular bond strength. (The higher the bond strength between molecules the higher the boiling point)
        \item There are three relevant inter-molecular bond forces (here ranked by strength from weakest to strongest)
        \begin{itemize}
            \item Van der Walls forces\hfill\break Occurs in non-polar bonds.\hfill\break non-polar molecules gain in bond strength the longer the "chain" of the molecule and the less "branched" it is.
            \item The Dipol-Dipol Force\hfill\break Occurs in polar bonds\hfill
            \item Hydrogen Bond ( Wasserstoffbrücken )\hfill\break Occurs in polar bonds involving a Hydrogen Atom (Overrules the Dipol-Dipol Force)
        \end{itemize}
    \end{itemize}
    \item 3. SP3 Hybridisierung
    \begin{itemize}
        \item the S1 Orbital can be combined with the P Orbital to form the SP3 Orbital, which is on a single energy level.
    \end{itemize}
\end{itemize}
\clearpage
Examples:\hfill
\begin{itemize}
    \item A:  \chemfig{C-C(-[2]H)(-[6]H)-\charge{90=\|,270=\|}{O}-H}\hspace{1cm}
    \begin{itemize}
        \item A Has 3 Non Polar Bonds ( C-C, C-H, C-H )\hfill\break and 3 Polar Bonds ( C-O, C-H ) and is therefore Non-Polar.
    \end{itemize}
    \item B:  \chemfig{H-C(-[2]H)(-[6]H)-C(-[2]H)(-[6]H)-C(-[2]H)(-[6]H)-O-H}
    \begin{itemize}
        \item B has 7 Non Polar Bonds ( C-H * 7 )\hfill\break and 2 Polar Bonds ( C-O, O-H ) is therefore Non-Polar. ( more than A)
    \end{itemize}
    \item C:  \chemfig{H-C(-[2]H)(-[6]H)-C(-[2]H)(-[6]H)-C(-[2]H)(-[6]H)-H}
    \begin{itemize}
        \item C has 10 Non Polar Bonds ( C-H * 7 ) and 0 Polar Bonds and is therefore exclusively Non-Polar.
    \end{itemize}
    \item D:  \chemfig{H-C(-[2]H)(-[6]H)-C(-[2]H)(-[6]H)-C(-[2]H)(-[6]H)-C(-[2]H)(-[6]H)-C(-[2]H)(-[6]H)-C(-[2]H)(-[6]H)-H}
    \begin{itemize}
        \item D  has 19 Non Polar Bonds ( C-C * 5, C-H * 14) and is therefore exclusively Non-Polar.
    \end{itemize}
    \item E:  \chemfig{H-C(-[2]H)(-[6]H)--C(-[2]C(-[0]H)(-[4]H))(-[6]C(-[0]H)(-[4]H))--C(-[2]H)(-[6]H)-H}
    \begin{itemize}
        \item E  has 19 Non Polar Bonds ( C-C * 5, C-H * 14) and is therefore exclusively Non-Polar.\hfill\break It however has a lower boiling point than item D, because the Van der waals bonds are weaker when more "nested".
    \end{itemize}


\chemfig{H-C(-[2]H)(-[6]H)-C(-[2]H)(-[6]H)-C(-[2]H)(-[6]H)-C(-[2]H)(-[6]H)-C(-[2]H)(-[6]H)-C(-[2]H)(-[6]H)-H}
\end{itemize}
\clearpage
\subsection{Lesson 2 - 28.09.2022}
Organic Chemistry\hfill\break
What is Organic Chemistry\hfill\break
\begin{itemize}
    \item Old - The Chemistry of Alive Stuff
    \item New - The Chemistry of Carbon including some exceptions
\end{itemize}
Hydrocarbons\hfill\break
\begin{itemize}
    \item Saturated Hydrocarbons
    \begin{itemize}
        \item Just single bonds
    \end{itemize}
    \item Unsaturated Hydrocarbons
    \begin{itemize}
        \item Allows all bonds
    \end{itemize}
\end{itemize}
Alkane
\begin{itemize}
    \item Methane - C1H4
    \item \chemfig{H-C(-[2]H)(-[6]H)-H}
    \item Ethane - \ce{C2H6}
    \item \chemfig{H-C(-[2]H)(-[6]H)-C(-[2]H)(-[6]H)-H}
    \item Propane - \ce{C3H8}
    \item \chemfig{H-C(-[2]H)(-[6]H)-C(-[2]H)(-[6]H)-C(-[2]H)(-[6]H)-H}
    \item Butane - \ce{C4H10}
    \item \chemfig{H-C(-[2]H)(-[6]H)-C(-[2]H)(-[6]H)-C(-[2]H)(-[6]H)-C(-[2]H)(-[6]H)-H}
    \clearpage
    \item Pentane - \ce{C5H12}
    \item \chemfig{H-C(-[2]H)(-[6]H)-C(-[2]H)(-[6]H)-C(-[2]H)(-[6]H)-C(-[2]H)(-[6]H)-C(-[2]H)(-[6]H)-H}
    \item Hexane - \ce{C6H14}
    \item \chemfig{H-C(-[2]H)(-[6]H)-C(-[2]H)(-[6]H)-C(-[2]H)(-[6]H)-C(-[2]H)(-[6]H)-C(-[2]H)(-[6]H)-C(-[2]H)(-[6]H)-H}
    \item Heptane - \ce{C7H16}
    \item \chemfig{H-C(-[2]H)(-[6]H)-C(-[2]H)(-[6]H)-C(-[2]H)(-[6]H)-C(-[2]H)(-[6]H)-C(-[2]H)(-[6]H)-C(-[2]H)(-[6]H)-C(-[2]H)(-[6]H)-H}
    \item Octane - \ce{C8H18}
    \item \chemfig{H-C(-[2]H)(-[6]H)-C(-[2]H)(-[6]H)-C(-[2]H)(-[6]H)-C(-[2]H)(-[6]H)-C(-[2]H)(-[6]H)-C(-[2]H)(-[6]H)-C(-[2]H)(-[6]H)-C(-[2]H)(-[6]H)-H}
    \item Nonane - \ce{C9H20}
    \item \chemfig{H-C(-[2]H)(-[6]H)-C(-[2]H)(-[6]H)-C(-[2]H)(-[6]H)-C(-[2]H)(-[6]H)-C(-[2]H)(-[6]H)-C(-[2]H)(-[6]H)-C(-[2]H)(-[6]H)-C(-[2]H)(-[6]H)-C(-[2]H)(-[6]H)-H}
    \item Decane - \ce{C10H22}
    \item \chemfig{H-C(-[2]H)(-[6]H)-C(-[2]H)(-[6]H)-C(-[2]H)(-[6]H)-C(-[2]H)(-[6]H)-C(-[2]H)(-[6]H)-C(-[2]H)(-[6]H)-C(-[2]H)(-[6]H)-C(-[2]H)(-[6]H)-C(-[2]H)(-[6]H)-C(-[2]H)(-[6]H)-H}
\end{itemize}
\clearpage
\noindent Drawing Conventions:\hfill\break
Example: Pentane
\begin{itemize}
    \item \chemfig{H-C(-[2]H)(-[6]H)-C(-[2]H)(-[6]H)-C(-[2]H)(-[6]H)-C(-[2]H)(-[6]H)-C(-[2]H)(-[6]H)-H}
    \item \chemfig{CH3-CH2-CH2-CH2-CH3}
    \item \chemfig{-[1]-[-1]-[1]-[-1]}
\end{itemize}
Isomers \hfill\break
Example: Hexane
\begin{itemize}
    \item \chemfig{-[1]-[-1]-[1]-[-1]-[1]}
    \item \chemfig{-[1](-[2])-[-1]-[1]-[-1]}
    \item \chemfig{-[1]-[-1](-[6])-[1]-[-1]}
    \item \chemfig{-[1](-[2])-[-1](-[6])-[1]-[-1]}
    \item \chemfig{-[1](-[2])(-[6])-[-1]-[1]-[-1]}
\end{itemize}
\clearpage
\subsection{Lesson 3 - 05.10.2022}
Benennung von Alkanen mit IUPAC

\noindent Rezept:
\begin{itemize}
    \item Längste C-Kette gibt Stammnamen
    \item Nummerieren der Stammkette
    \item Lage der Seitenketten durch vorherige Nummern angeben
    \item Länge der Seitenkette: Stamm + yl
    \item Bei mehreren gleichen Seitenketten: Di-, Tri-, Tetra-, Penta-, …
    \item Seitenketten alphabetisch nach Stamm eingeben
    \item Stammname hinten hinschreiben
\end{itemize}
\noindent Rechtschreibregeln:
\begin{itemize}
    \item Zwischen zwei Zahlen: Beistrich
    \item Zwischen Zahl und Buchstabe: Bindestrich
    \item Erster Buchstabe groß, alle anderen klein
    \item Wörter zusammenschreiben
\end{itemize}
\clearpage
Beispiel:\hfill\break\vspace{2cm}
\chemfig{-[1]-[-1]-[1](-[-6]-[3])(-[-2]-[-3])-[-1](-[-2.25])(-[-6])-[-7](-[-2])(-[-6]-[1]-[-6])-[-1](-[1]-[-1])(-[-3])-[-2](-[0])-[-3](-[-2])-[2.75]}

\noindent 1. - Längste C-Kette gibt Stammnamen.\hfill\break
\noindent Hier ist die längste Kette ein Decan, und bildet somit den Stamm des Moleküls.\hfill\break

\noindent 2. Nummerieren der Stammkette\vspace{1cm}

\chemfig{10-[1]9-[-1]8-[1]7(-[-6]-[3])(-[-2]-[-3])-[-1]6(-[-2.25])(-[-6])-[-7]5(-[-2])(-[-6]-[1]-[-6])-[-1]4(-[1]-[-1])(-[-3])-[-2]3(-[0])-[-3]2(-[-2]1)-[2.75]}

\clearpage
\noindent 3. - Lage der Seitenketten durch vorherige Nummern angeben.\hfill\break
\noindent 4. - Länge der Seitenkette: Stamm + yl.\hfill\break

\begin{center}
    \begin{tabular}{c|c}
         N C-Atom & Kette  \\
         \hline
         2 & Methyl \\
         \hline
         3 & Methyl \\
         \hline
         4 & Methyl \\
         4 & Ethyl \\
         \hline
         5 & Methyl \\
         5 & Propyl \\
         \hline
         6 & Methyl \\
         6 & Methyl \\
         \hline
         7 & Ethyl \\
         7 & Ethyl \\
         \hline
    \end{tabular}
    
    \vspace{2cm}
    
    \begin{tabular}{c|c}
        Kette & Anzahl \\
        \hline
        Methyl & 6 \\
        Ethyl & 3 \\
        Propyl & 1 \\
    \end{tabular}
\end{center}


\noindent Bei mehreren gleichen Seitenketten: Di-, Tri-, Tetra-, Penta-, …:\hfill\vspace{1cm}\break

\noindent 6 Methyl-Ketten $\implies$ Hexamethyl\hfill\break
\noindent 3 Ethyl-Ketten $\implies$ Triethyl\hfill\break
\noindent 1 Propyl-Kette $\implies$ Propyl\hfill\vspace{5mm}\break


\noindent Seitenketten alphabetisch nach Stamm eingeben:\hfill\vspace{5mm}\break
\noindent der Hex \& Tri Prefix ist nicht teils des Stammes und wird somit nicht im Sortieren betrachtet.\hfill\vspace{5mm}\break

\noindent Lösung:\hfill\vspace{5mm}\break
\noindent $4,7,7-Triethyl-2,3,4,5,6,6-Hexamethyl-5-Propyldecan$\hfill\break

\clearpage
\noindent Ringförmige Alkane\hfill\break

\noindent\begin{tabular}{c|c|c}
    Structure & Composition & Name  \\
    \hline
    \noalign{\vskip 2mm}    
    \chemfig{[:30]*3(---)}& \ce{C3H6} & Cyclopropan \\
    \hline
    \noalign{\vskip 2mm}    
    \chemfig{*4(----)} & \ce{C4H8} & Cyclobutan \\
    \hline
    \noalign{\vskip 2mm}    
    \chemfig{[:18]*5(-----)} & \ce{C5H10} & Cyclopentan \\
    \hline
    \noalign{\vskip 2mm}    
    \chemfig{[:30]*6(------)} & \ce{C6H12} & Cyclohexan \\
    \hline
\end{tabular}

\vskip 1cm

\noindent Cyclohexan liegt nicht flach und hat in dem 3-Dimensionalen Raum zwei verschiedene Formen:\hfill\break
\begin{itemize}
    \item Sesselform (am energieärmsten deswegen die häufigste und stabilste)
    %\item \chemfig[bond join=true]{-[:50]-[:-10]-[:10]-[:-130,,,,line width=1pt]-[:170,,,,line width=1pt]-[:190,,,,line width=1pt]}
    \item Halbsesselform
    \item Twist
    \item Wanne
\end{itemize}

\vskip 1cm

\noindent Hier eine Illustration, welche die Struktur der Formen zeigt:\hfill\break
\includegraphics[width=36.3mm, height=33.65mm]{cyclohexan.png}
\vskip 1cm
\noindent Hier eine Illustration, welche die Energie der Formen zeigt:\hfill\break
\includegraphics[width=58mm, height=40mm]{cyclohexan_energy.png}
\end{document}
