\documentclass{article}
\usepackage[utf8]{inputenc}
\usepackage{chemfig}
\usepackage{layout}

\title{Chem 2}
\author{Etienne Herzig}
\date{21. September 2022}

\begin{document}
\setlength{\textheight}{680pt}

\maketitle
\clearpage
\section{Overview}
\subsection{Grading}
Only the better two tests get counted and make up 80\% of the grade.
Homework \& Assignments should be submitted within online within two weeks and make up 20\% of the grade.
A bonus of ~4\% can be earned with consistent presence
\subsection{Test Dates}
\date{9.11.2022}\hfill\break
\date{14.12.2022}\hfill\break
\date{25.01.2023}\hfill\break
\clearpage
\section{Lessons}
\subsection{Lesson 1 - 21.09.2022}
Revision of principles learned in Chem 1, which bear relevance in Chem 2. \break
Important things to remember: \hfill
\begin{itemize}
    \item 1. Distinction Atom / Molecule
    \item 2. Polar / Non Polar
    \begin{itemize}
        \item A molecule is polar if the majority of bond in said molecule are polar. A bond is polar if one Atom in the bond has a higher electronegativity.
        \item Important Polar Bonds
        \begin{itemize}
            \item C - O
            \item O - H
            \item N - H
        \end{itemize}
        \item Important Non-Polar Bonds
        \begin{itemize}
            \item C - C
            \item C - H
        \end{itemize}
        \item Boiling Points are related to the intermolecular bond strength. (The higher the bond strength between molecules the higher the boiling point)
        \item There are three relevant intermolecular bond forces ( here ranked by strength from weakest to strongest )
        \begin{itemize}
            \item Van der Walls forces\hfill\break Occurs in unpolar bonds.\hfill\break Unpolar molecules gain in bond strength the longer the "chain" of the molecule and the less "branched" it is.
            \item The Dipol-Dipol Force\hfill\break Occurs in polar bonds\hfill
            \item Hydrogen Bond ( Wasserstoffbrücken )\hfill\break Occurs in polar bonds involving a Hydrogen Atom (Overrules the Dipol-Dipol Force)
        \end{itemize}
    \end{itemize}
    \item 3. SP3 Hybridisierung
    \begin{itemize}
        \item the S1 Orbital can be combined with the P Orbital to form the SP3 Orbital which is on a single energy level.
    \end{itemize}
\end{itemize}
\clearpage
Examples:\hfill
\begin{itemize}
    \item A:  \chemfig{C-C(-[2]H)(-[6]H)-\charge{90=\|,270=\|}{O}-H}\hspace{1cm}
    \begin{itemize}
        \item A Has 3 Non Polar Bonds ( C-C, C-H, C-H )\hfill\break and 3 Polar Bonds ( C-O, C-H ) and is therefore Non-Polar.
    \end{itemize}
    \item B:  \chemfig{H-C(-[2]H)(-[6]H)-C(-[2]H)(-[6]H)-C(-[2]H)(-[6]H)-O-H}
    \begin{itemize}
        \item B has 7 Non Polar Bonds ( C-H * 7 )\hfill\break and 2 Polar Bonds ( C-O, O-H ) is therefore Non-Polar. ( more than A)
    \end{itemize}
    \item C:  \chemfig{H-C(-[2]H)(-[6]H)-C(-[2]H)(-[6]H)-C(-[2]H)(-[6]H)-H}
    \begin{itemize}
        \item C has 10 Non Polar Bonds ( C-H * 7 ) and 0 Polar Bonds and is therefore exclusivly Non-Polar.
    \end{itemize}
    \item D:  \chemfig{H-C(-[2]H)(-[6]H)-C(-[2]H)(-[6]H)-C(-[2]H)(-[6]H)-C(-[2]H)(-[6]H)-C(-[2]H)(-[6]H)-C(-[2]H)(-[6]H)-H}
    \begin{itemize}
        \item D  has 19 Non Polar Bonds ( C-C * 5, C-H * 14) and is therefore exclusivly Non-Polar.
    \end{itemize}
    \item E:  \chemfig{H-C(-[2]H)(-[6]H)--C(-[2]C(-[0]H)(-[4]H))(-[6]C(-[0]H)(-[4]H))--C(-[2]H)(-[6]H)-H}
    \begin{itemize}
        \item E  has 19 Non Polar Bonds ( C-C * 5, C-H * 14) and is therefore exclusivly Non-Polar.\hfill\break It however has a lower boiling point than item D, because the Van der waals bonds are weaker when more "nested".
    \end{itemize}
\end{itemize}
\end{document}